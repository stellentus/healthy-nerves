\documentclass[12pt]{article}
\fontfamily{ptm}\selectfont

\usepackage{setspace} \doublespacing
\usepackage[margin=1.2in]{geometry}  % set the margins
\usepackage{graphicx}              % to include figures
\usepackage{epstopdf}
\usepackage{amsmath}               % great math stuff
\usepackage{amsfonts}              % for blackboard bold, etc
\usepackage{amsthm}                % better theorem environments
\usepackage{titlesec}

\linespread{1}

\usepackage{caption}
\captionsetup[figure]{labelsep=space}

\usepackage{authblk}
\title{Batch Effects Summary}
\author{James M Bell}
\renewcommand\Affilfont{\itshape\small}

\begin{document}

\maketitle

\section*{Introduction and Methods}

The data comes from three electrodiagnostic nerve excitability test (NET) datasets (median nerve, i.e. arm) from Canada (n=120), Japan (n=85), and Portugal (n=42). Additionally, some trials used a Canadian common peroneal nerve (i.e. leg) dataset (n=121) and a rat dataset (n=49). The data as analyzed here consists of 30 real-valued measures (age, temperature, and 28 excitability variables) and one categorical measure (male/female) extracted from the NET datasets.

In all of the following tests, two or more datasets were combined. A customized version of variation of information (BVI) was calculated based on a random sample of 80\% of the combined dataset, reported as mean (standard deviation) across 30 different random samples. A different random seed was used for each of the 30 trials, but the same 30 seeds were used for each test in a given figure.

Variation of information (VI) can be used to compare two clusterings (or known labels and a clustering) to measure how similar they are. Its minimum value is 0 and its maximum value is $2\log_2(n)$, where $n$ is the number of clusters. This dependence on $n$ makes it difficult to compare between groups. To account for this, all VOI scores were normalized by $2\log_2(n)$. They were then subtracted from 1 to give the final equation
\begin{equation}
BVI(x,y) = 1-\frac{VI(x,y)}{2\log_2(n)},
\end{equation}
where BVI is the batch-effect variation of information for the two clusterings $x$ and $y$ with the number of clusters fixed to $n$. A score of 0 indicates no similarity between clusters, while 1 indicates they are identical. Therefore, BVI for the combined Canadian/Japanese/Portuguese dataset should be near 0 if it is appropriate to combine them into a normative dataset, while any combination with leg or rat data should be significantly higher. BVI is normalized to account for different numbers of clusters, so it is appropriate to compare it even when the number of groups is different.

The data was normalized before the BVI was calculated by combining the data (across all labels) and converting it to zero mean, unit variance.

\emph{My doubts and concerns are emphasized like this.}

\pagebreak

\section*{Results and Some Discussion}

\begin{figure*}[ht]
  \centering
       \includegraphics[width=\textwidth]{../img/batch-norm-rand.png}
         \caption{}
  \label{fig:norm-rand}
\end{figure*}

In Figure \ref{fig:norm-rand}, BVI for the \textbf{Normative Data} (i.e. combined Canadian, Japanese, and Portuguese data) is compared to two different randomly generated lists of three labels with a length of 247 (\textbf{Random Labels}). (BVI for identical lists is always exactly 1, so it is not shown.) As expected, BVI for random data is near zero. These are also compared to \textbf{Shuffled Normative} data, which shows a BVI score when the labels for the normative data are scrambled. Since this BVI is higher than the random labels, there must be something about sorting the data that increases its BVI score. This will be investigated further in Figure \ref{fig:country-splits}.

\pagebreak

\begin{figure*}
  \centering
       \includegraphics[width=\textwidth]{../img/batch-country-splits.png}
         \caption{}
  \label{fig:country-splits}
\end{figure*}

In Figure \ref{fig:country-splits}, each contry's data is randomly split into three equal sub-groups and compared to random data of the same size. For example, the 150 Canadian samples were split into three 50-sample groups to test if this homogeneous data shows batch effects. A comparison of the first and second results in this figure shows that the batch effects are larger than three groups of 50 random labels. This is likely because the 80\% random-sampling of three groups of 50 samples results in three groups with sizes close to 40, while there is much higher variance in the splits of the Canadian data. The clustering algorithm might split the 120 Canadian samples (from 80\% of 150) into three groups of 40, or it might split in an unbalanced way, like 20-40-60. As a result, while the BVI calculated for random data provides a lower limit on our expectation, it is not realistic to expect such a low BVI. The results for \textbf{Shuffled Normative} (second-last in Figure \ref{fig:country-splits}, but described in Figure \ref{fig:norm-rand}) provide a more realistic target. However, that BVI is still lower than the score for the normative data.

\pagebreak

\begin{figure*}
  \centering
       \includegraphics[width=\textwidth]{../img/batch-group-size.png}
         \caption{}
  \label{fig:group-size}
\end{figure*}

All of the results in Figures \ref{fig:norm-rand} and \ref{fig:country-splits} used three labels, but future figures will not. While BVI theory allows for comparison of BVI with different numbers of clusters, it is illustrative to present a visual comparison of diverse cluster sizes (Figure \ref{fig:group-size}). This set of tests generated 60 labels for each class, ranging from 2 to 6 classes. Then 80\% of the labels were sampled 30 times. There is a small increase in BVI as the number of clusters increases, but this increase is minor compared to the BVI for the normative data.

\emph{I'm not sure why this is happening, but I have a partial understanding. These should all be clustered near the minimum (0), but negative values are impossible, so the mean will necessarily be above zero. I don't know how to do the math to predict the mean BVI (or VOI) for a random variable, but I suspect it shows this small increase from zero.}

\pagebreak

\begin{figure*}
  \centering
       \includegraphics[width=\textwidth]{../img/batch-rc.png}
         \caption{}
  \label{fig:rc}
\end{figure*}

To be confident in the normative data it is necessary to show a near-zero BVI, but that is not sufficient. The BVI must also change significantly when non-normative data is added. To test the impact of widespread changes, 6 of the 28 excitability variables were subtly adjusted (see Figure \ref{fig:rc}). In the ``RC shift'' tests, the 7 variables derived from one portion of the NET (recovery cycle, RC) were transformed in one of the three datasets as if the corresponding underlying data had shifted left or right, imitating a plausible technical difference between datasets. However, only 3 of those variables change appreciably, so this adjustment does not change the data very much. In contrast, in the ``RC shrink'' tests, the same 7 variables were all decreased by 50\%. The results show an increase in BVI.

\emph{In a publication, show a graph of RC with these changes.}

% Maybe in my thesis I can use this as a point about updating the list of excitability indices. Can I suggest in the future a reason or way to dispose of the excitability measures in favor of the entire waveform (or I think a different set from feature selection). As part of this I could show two people with identical excitability measures, but with different waveforms after 50ms.

\pagebreak

\begin{figure*}
  \centering
       \includegraphics[width=\textwidth]{../img/batch-vs-legs.png}
         \caption{}
  \label{fig:vs-legs}
\end{figure*}

Testing with simulated batches shows some minor increases to BVI. It will also be illustrative to consider batch effects between the normative data and some other datasets. Leg data is similar to arm data, but it is different enough that it should show up as a different batch. In Figure \ref{fig:vs-legs}, the the Canadian median nerve (arm) data is replaced with with CP nerve (leg) data (labeled as \textbf{Can Arms $\rightarrow$ Legs}). This causes a significant BVI increase, as expected. Adding the leg data (\textbf{Add Legs}) results in 4 groups. Now one quarter of the data is from legs instead of one third as in the previous test, so it is expected to calculate an BVI score lower than the last trial but higher than the normative data. Finally, comparing all of the normative data (from all three countries labelled together) to the leg data (\textbf{Normative vs Legs}) shows the legs are obviously a different batch.

\pagebreak

\begin{figure*}
  \centering
       \includegraphics[width=\textwidth]{../img/batch-vs-rats.png}
         \caption{}
  \label{fig:vs-rats}
\end{figure*}

Rat data is radically different from human, so it should be even more clearly a different batch than a comparison to leg data. This rat data is heterogeneous, consisting of measurements from three different muscles (tibialis anterior, soleus, and tail) under the effects of two different anaesthetics (sodium pentobarbital and ketamine xylazine) in two different clinical conditions (healthy and spinal cord injury). All of these types of rat data have been collected together under a single label. Figure \ref{fig:vs-rats} compares BVI with human median and rat data. Replacing the Canadian or Japanese data with rat data causes a large increase in BVI. Adding rat data to the three normative groups has a slightly smaller impact on BVI (since rat data is now one of four groups instead of one of three groups). Finally, when the normative human median data is compared to the diverse rat data, the BVI score approaches its maximum value, indicating a clear difference between species (in spite of the heterogeneity of the rat data).

\pagebreak

\section*{More Discussion and Conclusions}

Figures \ref{fig:rc}–\ref{fig:vs-rats} show that BVI is an effective measure for detecting batch effects. The normative data BVI was 0.13$\pm$0.04. When compared to legs, it doubled to 0.29$\pm$0.08. When compared to rats, it increased to 0.99$\pm$0.01. This indicates that normative human arm data is somewhat distinguishable from leg data and very distinguishable from rat data, so BVI can be used to determine if NET data from diverse sources can be appropriately combined. BVI is sensitive to large changes (e.g. leg or rat), though it may not be sensitive to smaller, biologically-plausible changes (e.g. refractory period). These results suggest that there are no large batch effects between the Canadian, Japanese, and Portuguese data. If batch effects exist, they must be small.

\emph{I suspect if the majority of the parameters changed by 50\%, I wouldn't detect it, so I'm not sure this really counts as ``small''.}

\end{document}
